% !TEX TS-program = pdflatex
% !TEX encoding = UTF-8 Unicode

% This is a simple template for a LaTeX document using the "article" class.
% See "book", "report", "letter" for other types of document.

\documentclass[11pt]{article} % use larger type; default would be 10pt

\usepackage[utf8]{inputenc} % set input encoding (not needed with XeLaTeX)

%%% Examples of Article customizations
% These packages are optional, depending whether you want the features they provide.
% See the LaTeX Companion or other references for full information.

%%% PAGE DIMENSIONS
\usepackage{geometry} % to change the page dimensions
\geometry{a4paper} % or letterpaper (US) or a5paper or....
% \geometry{margin=2in} % for example, change the margins to 2 inches all round
% \geometry{landscape} % set up the page for landscape
%   read geometry.pdf for detailed page layout information

\usepackage{graphicx} % support the \includegraphics command and options

% \usepackage[parfill]{parskip} % Activate to begin paragraphs with an empty line rather than an indent

%%% PACKAGES
\usepackage{booktabs} % for much better looking tables
\usepackage{array} % for better arrays (eg matrices) in maths
\usepackage{paralist} % very flexible & customisable lists (eg. enumerate/itemize, etc.)
\usepackage{verbatim} % adds environment for commenting out blocks of text & for better verbatim
\usepackage{subfig} % make it possible to include more than one captioned figure/table in a single float
% These packages are all incorporated in the memoir class to one degree or another...
\usepackage{amsmath}
%%% HEADERS & FOOTERS
\usepackage{fancyhdr} % This should be set AFTER setting up the page geometry
\pagestyle{fancy} % options: empty , plain , fancy
\renewcommand{\headrulewidth}{0pt} % customise the layout...
\lhead{}\chead{}\rhead{}
\lfoot{}\cfoot{\thepage}\rfoot{}

%%% SECTION TITLE APPEARANCE
\usepackage{sectsty}
\allsectionsfont{\sffamily\mdseries\upshape} % (See the fntguide.pdf for font help)
% (This matches ConTeXt defaults)

%%% ToC (table of contents) APPEARANCE
\usepackage[nottoc,notlof,notlot]{tocbibind} % Put the bibliography in the ToC
\usepackage[titles,subfigure]{tocloft} % Alter the style of the Table of Contents
\renewcommand{\cftsecfont}{\rmfamily\mdseries\upshape}
\renewcommand{\cftsecpagefont}{\rmfamily\mdseries\upshape} % No bold!

%%% END Article customizations

%%% The "real" document content comes below...

\title{Rendu ASR 2}
\author{Joël Felderhoff}
%\date{} % Activate to display a given date or no date (if empty),
         % otherwise the current date is printed 

\begin{document}
\maketitle

\section{Boutisme}
\subsection{}
L'endianess se concentre sur l'ordre des octets dans la mémoire : Big indian met les octets de poids fort sur les adresses les plus petites, Little indian met les octets de poids fort sur les adresses les plus grosses. Il n'y a rien à ordonner si on a qu'un octet, donc pas besoin de faire 2 versions pour uint8 (1 octet)

\setcounter{section}{2}
\section{Un peu de maths}
\subsection{}
La structure fat32\_driver indique le nombre de secteurs réservés : ``nb\_reserved\_sectors''. Les secteurs réservés étant situés au début du fichier, on peut se placer après simplement.
\subsection{}
La position dans l'image disque du début de la FAT est donnée par 
$$\text{offset} = \text{nb\_1er\_secteur} \cdot \text{nb\_bytes\_per\_sector} $$
Alors la position dans le fichier de la $n$ième entrée de la FAT sera donnée par :
$$ \text{position} = \text{offset} + 4n$$
Toutes les valeurs sont données en octets
\setcounter{subsection}{3}
\subsection{}
On a tout d'abord un offset (en secteur) égal au nombre de secteurs réservés plus le nombre de secteurs utilisés par les FATs, ensuite la formule tombe toute seule :
$$
  \text{first\_sect\_cluster}(C) = \text{nb\_sect\_reservés} + \text{nb\_sect\_per\_fat} \cdot \text{nb\_fat} + (C-2) \cdot \text{nb\_sector\_per\_cluster}
$$


\section*{Notes}
J'utilise emacs pour coder, ce qui fait que mon indentation est plus petite que celle des fichiers donnés dans l'archive, j'ai donc tout réindenté

\end{document}
